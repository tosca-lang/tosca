In this section we describe the programming language \Tosca, starting
from its basis in higher-order rewriting and leading to its actual features.
%
In Section~\ref{sec:CRS} we describe the theory behind \Tosca:
higher-order term rewriting, in particular Combinatory Reduction
Systems. 
Section~\ref{sec:CRStocore} gives an overview over the
\Tosca Core grammar and relates it to CRS. Then
Section~\ref{sec:coretotransscript} shows how \Tosca Core 
is sweetened for human consumption.

\subsection{Combinatory Reduction Systems} \label{sec:CRS}

Rewriting naturally underlies many, if not all tasks of a standard
compiler. Hence it is naturally to encode a compiler 
generator in a rewriting-based language, an idea which was
taken up in \crsx~\cite{2010_rose, 2011_rose}.
 In the case of \Tosca the foundational basis is given by \emph{Combinatory
  Reduction Systems}~(\emph{CRSs} for short)
\cite{Aczel:78,1980_klop}.  CRSs form a formalism for higher-order
rewriting, that is, terms may contain $\lambda$-abstractions (aka
closures) and bound variables, and we may rewrite such terms.

In contrast to \emph{higher-order rewrite systems} (HRSs for short)~\cite{1998_mayr-nipkow} 
which form the formal basis of the interactive theorem prover \Isabelle, CRSs are
a \emph{second}-order formalism and are untyped.
Intuitively that means that a variable can represent a
first-order but not a higher-order function. For example, a variable
can represent the function $\plus$ but not the $\map$ function in
Example~\ref{ex:map}.
This restriction, however, is not really crucial as HRSs and CRSs are simulation
equivalent~\cite{2003_van_raamsdonk}.

In the context of \Tosca, CRSs are a more natural formalism, as the 
rewriting machinary feels less heavy. In particular CRSs can be conceived
more naturally as a programming language. In passing we mention that
\emph{optimality} has been studied in the context of \emph{Interaction Systems}~\cite{AG:98},
which form a subclass of CRSs and thus forms a subclass of \Tosca Core.

We follow van Raamsdonk~\cite{2003_van_raamsdonk} in our presentation 
and restrict to the essentials required in context of this paper. The
reader is kindly refered to~\cite{2003_van_raamsdonk} for further details.
We start by introducing the main ingredient: meta terms. 
%
\begin{definition} \label{def:term} 
  Let $\V$ be a set of \emph{variables}, $\M$ a set of \emph{meta
    variables}, and $\F$ a set of \emph{function symbols}. Every
  $X \in \M$ and every $f \in \F$ has a fixed arity.
  %
  Let $x \in \V$, $X \in \M$, and $f \in \F$. A \emph{meta term} $t$
  is
  %
  \[ t := %
         x 
    \mid f( t_1, \ldots, t_n) 
    \mid X(t_1, \ldots, t_l)
  \mid [x] t
 \]
 %
 The expression $[x] t$ provides CRS's notation for $\lambda$-abstraction, in
 particular the variable $x$ is bound in $t$.
 A meta term without meta variables is called a \emph{term}.
\end{definition}
%
\newcommand*{\Yhd}{Y_{\text{hd}}}
\newcommand*{\Ytl}{Y_{\text{tl}}}
\newcommand*{\alphaeq}{=_\alpha}

Recall the formulation of the map function from Example~\ref{ex:map}. To
avoid confusion we make use of a new signature $\F \defsym \{ \map, \cons, \nil, \plus, \one \}$.
Furthermore, we need the meta variables $\M \defsym \{F, \Yhd, \Ytl, \ldots \}$, and 
the variables $\V \defsym \{ x, \ldots \}$. Then we can construct the meta term:
$\map( [x] F(x), \cons( \Yhd, \Ytl))$.

Meta terms are considered modulo $\alpha$-conversion, for example,
the meta terms $[x]F(x)$ and $[y] F(y)$ are identified.
One may think of bound variables as formal parameters, whose names do
not matter too much either. Furthermore, we tacitly assume the
variable convention~\cite{2003_van_raamsdonk}.

We define the root $\rt(t)$ of a meta term $t$ according
to the grammar given in Definition~\ref{def:term}. I.e., $\rt(x) \defsym x$,
$\rt(f( t_1, \ldots, t_n)) \defsym f$; $\rt(X(t_1, \ldots, t_l)) \defsym X$,
and finally $\rt([x] t) \defsym [x]$. 

Similarly the set of \emph{positions} of $t$ is defined as a string
of natural numbers to identify occurrences of meta terms in $t$.
%
\begin{equation*}
  \Pos(t) \defsym
  \begin{cases}
    \{\varepsilon\} 
          & \text{if $t = x$}\\
    \{\varepsilon\} \cup \{ip \mid p \in \Pos(t_i)\} 
          & \text{\parbox{20ex}{if $t = f(t_1, \ldots, t_n))$ or
            $t = X(t_1, \ldots, t_n)$}}\\
    \{\varepsilon\} \cup \{0p \mid p \in \Pos(t)\} 
          & \text{if $t = [x]t$}
          \tpkt    
  \end{cases}
\end{equation*}
%
For example with respect to $t \defsym \cons(F(\Yhd), \map([x]F(x), \Ytl))$,
the meta term $F(x)$ occurs at position $210$. The meta term occurring
at position $p$ in $t$ is denoted as $\atpos{t}{p}$.

Based on the definition of meta terms, we want to define rewrite
rules. Not every meta term is a good candidate to be the left hand
side of a rewrite rule. Fortunately, there is a sensible restriction:
\emph{patterns}.
%
\begin{definition}
  A meta term~$t$ is a \emph{pattern}, if for every meta variable
  $X(t_1, \ldots t_l)$ all arguments $t_i$ are distinct bound
  variables.
\end{definition}
%
This ``pattern restriction'' guarantees that unification with a term
is decidable. Note that second-order unification in general is undecidable~\cite{Goldfarb:81}.
Thus we can compute whether a ``rule matches with
term'' and if we can ``apply a rule''. The specifics behind
higher-order pattern unification will be given in
Section~\ref{sec:specification}, where we will present its implementation
in \Tosca.

To define a rewrite step, we also need the notion of contexts. A
context is just that: a context, which is not affected at all by a
rewrite step. To formally define contexts, we need a fresh function
symbol $\hole \not\in \F$. A term $C$ with exactly one occurrence of
$\hole$ is called a \emph{context}. For replacing~$\hole$ by a
term~$t$ in $C$ we write $C[t]$. It is important to note, that a
context may trigger variables to be bound, i.e., they will be
captured. Thus, $[x] f(\hole) \neq [y] f(\hole)$, as the variable $x$
will be bound in one, but not the other.
%
\begin{definition} \label{def:rule} \label{def:rewritesystem} 
  %
  A \emph{rule} $\ell \to r$ consists of two meta terms $\ell$ and
  $r$, where
  %
  \begin{enumerate*} [label=\itshape(\roman*)]
  \item \label{It:Fun}   $\ell$ is a pattern with root $f \in \F$,
  \item \label{It:Var}   all meta variables in $r$ occur in $\ell$, and 
  \item \label{It:Bound} all variables are bound. 
  \end{enumerate*}
  %
  A set of rewrite rules is a called a
  \emph{Combinatory Rewrite System}. 
\end{definition}
%
By Condition~\ref{It:Fun} the root of the left-hand side has to be a
\emph{function symbol}.
%
Condition~\ref{It:Var} guarantees that the meta variables $r$ are
defined in $\ell$, i.e., no new meta variable appears in
$r$. By Condition~\ref{It:Bound} rules do not contain variables, that are not
bound.
%
\begin{example}[continued from Examples~\ref{ex:map}] \label{ex:rules} %
  The $\map$ function can be defined with the following two rewrite
  rules.
  \begin{align*}
    & \map([x] F(x), \nil)  \to \nil \\
    & \map([x] F(x), \cons(\Yhd,\Ytl)) 
      \to \cons(F(\Yhd), \map([x]F(x), \Ytl))
      \tpkt
  \end{align*}
  %
  Assuming for example the unary representation of numbers via the symbols
  $\plus$ and $\one$ 
  and the encoding of lists via $\cons$ and $\nil$, 
  we can extend this CRS as we did in
  Example~\ref{ex:map:2} with a $\main$-function.
  %
  \begin{equation*}
    \main \to \map ([x] \plus(1,x), (1,2,3))
    \tpkt
  \end{equation*}
\end{example}

Now we have defined a rewrite system, i.e., a program. But
how to compute something, i.e., how to ``apply a rule to a term''?
First we think about how we can replace the meta variable $F$ with
$[x]\plus(1, x)$. For that we introduce the notion of \emph{substitutions},
\emph{substitutes} and \emph{valuations}.

The \emph{substitution} $s[\xs \defsym \ts]$ of a sequence of terms $\ts$ 
for a sequence of variables $\xs$, is inductively defined as follows.
%
\[ 
  %
  s[ \xs \df \ts ] \defsym
  %
  \begin{cases}
    t_i  
      & \text{if } s = x_i  \in \xs \\
    y    
      & \text{if } s = y \not\in \xs \\
    f( s_1[\xs \df \ts], \ldots, s_l[\xs \df \ts]) 
      & \text{if } s = f(s_1, \ldots, s_l) \\
    [y] s'[\xs \df \ts]
      & \text{if } s = [y] s' \wedge y \not\in \xs \\
  \end{cases}
  \]
%
For the last case it is important, that no variable becomes bound,
thus $ y \not\in \xs$. However, this is easily assumed making
tacit use of $\alpha$-conversion.

Let $s$ be a term, and $ x_1, \ldots, x_n$ be variables. 
The \emph{$n$-ary substitute}, denoted as $\ULAM x_1 \ldots x_n. s$,
is defined as follows:
%
\begin{equation*}
  (\ULAM x_1 \ldots x_n. s) \ts \defsym s[ \xs \df \ts ] \tpkt
\end{equation*}
%
Let $\sigma$ be a mapping that assign $n$-ary substitutes to $n$-ary meta variables
and let $t$ be a meta term.
A \emph{valuation} is the homomorphic extension of $\sigma$, inductively
defined as follows. 
% 
\[
t^\sigma \defsym
% 
\begin{cases}
  x  
    & \text{if } t = x \\
  f( t_1^\sigma , \ldots t_l^\sigma) 
    & \text{if } t = f(t_1, \ldots, t_l) \\
   [x] t'^\sigma 
    & \text{if } t = [x] t' \\
   \sigma(X)(t_1^\sigma, \ldots, t_k^\sigma)
     & \text{if } t = X(t_1^\sigma, \ldots, t_k^\sigma) \\
\end{cases}
\]
%
Applying the above definition is straight forward. In
particular $\sigma(X)$ means replace $X$ by its image. 

We illustrate the interplay of substitutions, substitutes and
valuations by an example.
%
\begin{example} \label{ex:val}
  For
  $ \sigma = \{ {F \mapsto \ULAM x. \plus(1,x)}, {\Yhd \mapsto 1},
  {\Ytl \mapsto \nil} \} $
  we get
  \begin{align*}
    & (\map( [x] F(x), \cons{\Yhd}, \Ytl)))^\sigma \\ 
    & =  \map( [x] (\ULAM x . \plus(1,x)) (x), \cons(1, \nil))\\
    & = \map([x] \plus(1,x), \cons(1,\nil))
      \tpkt
  \end{align*}
  Similarly, we get
  \begin{align*}
    & (\cons(F(\Yhd), \map( [u] F(u), \Ytl))^\sigma \\
    & = \cons((\ULAM x . \plus(1,x))(1), 
        \map( [x] (\ULAM x . \plus(1,x)) (x) , \nil))\\
    & = \cons(\plus(1,1), \map([x] \plus(1,x),\nil))
  \end{align*}
\end{example}

We emphasize that CRSs make use of $\lambda$-abstraction on two levels.
One the one hand one employs binders on the object level, where abstractions
are denoted as $[x]t$. On the other hand, binders are used on the meta level
in the form of substitutes, denoted as $\ULAM x . s$. Furthermore we note that
the interplay between substitutions, substitutes and valuations in CRSs amounts
to performing a complete $\beta$-development.
%
Finally, we can give a declarative definition of a rewrite step.
%
\begin{definition}
  A term $s$ \emph{rewrites} to a term $t$, denoted by $s \rw t$, if
  $s = C[\ell^\sigma]$ and $t = C[r^\sigma]$ for a context~$C$, a
  valuation~$\sigma$, and a rule $\ell \to r$.
\end{definition}

\begin{example} 
  We have the rewrite step 
  \begin{align*}
    & \map( [u] \sq(u), \cons(1,\nil)) \rw  \\
    &\ \ \cons(\sq(1), \map( [u] \sq(u), \nil)) 
  \end{align*}
  with $C= \hole$, $\sigma$ from Example~\ref{ex:val}, and the second
  rewrite rule from Example~\ref{ex:rules}.
\end{example}

Revisiting Example~\ref{ex:rules} and taking stock of the structure
of the rules, we see that the provided CRS is restricted
in three crucial points:
%
\begin{enumerate}
\item Each meta variable on the left-hand side occurs at most once.
\item None of the arguments of the left-hand side contains symbols
\emph{defined} by any of the other rules, that is, neither $\nil$
nor $\cons$ occur as root of a left-hand side.
\item The rules are not \emph{overlapping}, that is, the rules are non-ambigious.
\end{enumerate}

These observations motivate the next definitions. 
%
Let a CRS $\R$ over the signature $\F$ be given. Then $\R$ is called \emph{left-linear}, 
if every meta variable on the left-hand side of a rule occurs at most once.
Further, we can write the signature $\F$ as the disjoint set of
symbols $\DefSymb$ and $\ConSymb$, where $\ConSymb$ collects all symbols which 
do \emph{not} occur as root symbol of a left-hand side in $\R$. 
If all arguments of left-hand
side contain only symbols from $\ConSymb$, then $\R$ is called a \emph{constructor CRS}. 

Let $s$ and $t$ be meta terms. We say that $s$ \emph{overlaps} with $t$, 
if there exists a valuation $\sigma$ and a non-meta variable position 
$p$ in $s$ such that $\atpos{s}{p}^\sigma = t^\sigma$.
Two rules $l_1 \to r_1$ and $l_2 \to r_2$ are
\emph{overlapping} if $l_1$ overlaps with $l_2$ or vice versa and the overlap
does not occur at the root when two copies of the same rule are considered. Finally, if there
are no overlapping rules, then $\R$ is called \emph{non-overlapping}. 
%
Left-linear and non-overlapping CRSs, are called \emph{orthogonal}. 

If any successful evaluation yields the same result, i.e., 
we do not have to worry about non-determinism, we call a rewrite system 
\emph{confluent}, or \emph{Church-Rosser}. 
%
We will make use of the following result later on, cf.~Section~\ref{sec:specification}. 
%
\begin{theorem} \cite{1980_klop} 
\label{t:1}
  Orthogonal Combinatory Reduction Systems are \emph{confluent}.
\end{theorem}

As mentioned above, \emph{Interaction systems}, introduced by Asperti
et al.~\cite{AL:94,AL:96} form a subset of CRSs. More precisely
interaction systems form a subset of constructor, orthogonal CRSs and
are morally equivalent.

\subsection{\Tosca Core} \label{sec:CRStocore}

In this section we describe the programming language \Tosca, focusing
on its foundation in higher-order rewriting.
The \Tosca Core is a subset of the \Tosca language. The
idea behind \Tosca Core is to reduce the rich programming
language \Tosca to its bare necessities, which is in direct
correspondence to left-linear, constructor %based 
CRSs defined in the last section. 

Having a small core
makes it easier to reason about it, easier to process, and easier to
analyze automatically. So what is a necessity? We distinguish between
semantic and technical necessities. Semantic relates to the
expressibility of \Tosca, technical to the implementation.

The \antlr grammar of the core is available at
%
\anolink{\ToscaPath/src/core/Core.g4}
        {\cd{Core.g4}}{\cd{src/core/Core.g4}}.
%
As in Section~\ref{sec:CRS} we start with the definition of meta terms
or, as in \Tosca Core: \antlrIn{cterms}. Here, the \antlrIn{c}
stands for ``core''.%
%
\begin{lstANTLR}
cterm :  VARIABLE
       | CONSTRUCTOR cterms?
       | METAVAR cterms?  
       | '[' VARIABLE ']' cterm
       | literal

cterms : '(' cterm (COMMA cterm)* ')'
\end{lstANTLR}
%
A \antlrIn{cterm} looks similar to a meta term in
Definition~\ref{def:term}. The first four cases directly relate. A
\antlrIn{VARIABLE} corresponds to an element in $\V$, and starts with a
lower case letter, e.g., \ToscaIn{x}. Similarly, a
\antlrIn{CONSTRUCTOR}
corresponds to a function symbol in $\F$.%
\footnote{We emphasize that the keyword \antlrIn{CONSTRUCTOR} means function symbol
in the signature, rather than merely constructor as implicit in the above
notion of \emph{constructor} CRSs.}
%
In \Tosca function symbols start with upper case letters,
e.g., \ToscaIn{F}. Next, \antlrIn{METAVAR} indicates elements in
$\M$, which start with \antlrIn{\#}, e.g. \ToscaIn{\#X}. Finally,
the case in line~4 binds the \antlrIn{VARIABLE} in the
\antlrIn{cterm}, e.g., \antlrIn{[x]F(x)} corresponds to $ [x] f(x)$. The next case is a \antlrIn{literal}:
%
\begin{lstANTLR}
  literal : STRING
           | NUMBER
\end{lstANTLR}
%
Literals are built-in in \Tosca---purely for efficiency reasons,
i.e., it is a technical necessity. They are conceptually not difficult
and easily expressible in CRSs.

The arguments of \antlrIn{CONSTRUCTOR} and \antlrIn{METAVAR}
are \emph{optional} \antlrIn{cterms}. This is the implementation of
``dropping the parenthesis'' for function symbols without
arguments. That is, we usually write simply \ToscaIn{True} and
not \ToscaIn{True()}---which we would need to write, strictly according to
Definition~\ref{def:term}. 

The main concepts of CRS in Section~\ref{sec:CRS}---patterns, meta terms,
contexts---are directly transferable to \antlrIn{cterms}.

The \antlrIn{cterm}s are the basic building block of a \Tosca
program. Now we see, how to create a \Tosca Core program, that
is, a rewrite system.
%
\begin{lstANTLR}
  ctransscript : cdecl+

  cdecl : 'rule' cterm '→' cterm
          | 'enum' CONSTRUCTOR
          | ...
\end{lstANTLR}
%

Every \Tosca Core program consists of a sequence of declarations,
namely \antlrIn{cdecl}. One case \antlrIn{cdecl} is a rule
declaration, also indicated by the \ToscaIn{rule} keyword. Similarly
to Section~\ref{sec:CRS} above, a rule consists of two terms,
separated by \ToscaIn{'→'}. As in Definition~\ref{def:rule}
restrictions need to be imposed. However, due to practical concerns
some lesser constraints have been weakened. For example the following
\Tosca rule \ToscaIn{F(x) → x} is valid in \Tosca Core, but not as
CRS. The variable $x$ is free, while Definition~\ref{def:rule}(iii)
demands that $x$ is bound. This is a practical necessity. We need free
variables to be able to handle invalid embedded programs in which some
variable are used without declaring them.

But in \antlrIn{cdecl}, what is hidden behind \antlrIn{...} ? For one,
we have import declarations. This technicality is needed to generate
\java code, but not the focus of this section.  Also hidden behind
\antlrIn{...} are type declarations, whose full incorporation is work in
progress. We briefly report on this when we mention future work in
Section~\ref{sec:conclusion}.
%
One subtle, but severe difference is that in a \Tosca program,
the rules are given in a \emph{sequence} ie., in lexical order. In the CRS setting,
Definition~\ref{def:rewritesystem}, a program corresponds to a
\emph{rewrite system}, which is a \emph{set} of rules and does not
enforce any order on the rules. We will investigate this difference,
and the consequences, in more detail in
Section~\ref{sec:specification}.

\subsection{\Tosca Core To
  \Tosca} \label{sec:coretotransscript} 

{ 
\renewcommand{\t}[1]{\ToscaIn{#1}}
% production
\newcommand{\p}[1]{\emph{#1}}

\begin{figure*}[!t]
      \begin{tabular}{p{6cm} c p{6.5cm}}
      $\C$ ⟦ \p{decl}$_1$ ... \p{decl}$_n$ ⟧               & = & $\CD$⟦ \p{decl}$_1$ ⟧ ... $\CD$⟦ \p{decl}$_n$ ⟧                \\        % & (1) \\ 
      $\CD$⟦ \t{rule CONSTRUCTOR} \p{args?} → \p{terms} ⟧ & = & \t{rule CONSTRUCTOR} $\CAS$⟦ \p{args?} ⟧ → $\CT$⟦ \p{terms}⟧  \\       % & (2) \\ 
      $\CD$⟦ \t{enum CONSTRUCTOR} ⟧                        & = & \t{enum CONSTRUCTOR}   				        \\ [3pt] % & (3) \\ 
      %
      $\CT$⟦ \t{CONSTRUCTOR} \p{args?} ⟧                 & = & \t{CONSTRUCTOR} $\CAS$⟦ \p{args?} ⟧                            \\ % & (4) \\  
      $\CT$⟦ \t{()} ⟧                                    & = & \t{Nil}				                              \\ % & (5) \\
      $\CT$⟦ \t{(} \p{term} \t{)} ⟧                     & = & $\CT$⟦ \p{term}  ⟧                                            \\ % & (6)  \\   
      $\CT$⟦ \t{(} \p{term}$_1$ \t{,} \p{term}$_2$ ... \t{,} \p{term}$_n$ \t{)} ⟧ & = %
                                                         & \t{Cons(} $\CT$⟦ \p{term}$_1$ ⟧ \t{,} $\CT$⟦ \p{term}$_2$ ... \p{term}$_n$ ⟧ \t{)}  \\%  & (7) \\  
      %
      $\CT$⟦ \t{VARIABLE} ⟧          & = & \t{VARIABLE}                    \\       % & (8)  \\                       
      $\CT$⟦ \t{STRING} ⟧            & = & \t{STRING}                      \\       % & (9)  \\                               
      $\CT$⟦ \t{NUMBER} ⟧            & = & \t{NUMBER}                      \\       % & (10) \\                      
      $\CT$⟦ \t{METAVAR} \p{margs?} ⟧ & = & \t{METAVAR} $\CMS$⟦ \p{margs?} ⟧  \\       % & (11) \\                       
      $\CT$⟦ \p{concrete} ⟧          & = & $\CC$⟦ \p{concrete} ⟧           \\ [3pt] % & (12) \\
      %
      $\CAS$⟦ \t{()}⟧                                              & = & \t{()}                                                             \\       % & (13) \\
      $\CAS$⟦ \t{(} \p{scope}$_1$ ...  \t{,} \p{scope}$_n$ \t{)}⟧   & = & \t{(} $\CS$⟦ \p{scope}$_1$⟧ ... \t{,} $\CS$⟦ \p{scope}$_n$⟧ \t{)}  \\ [3pt] % & (14) \\
      %
      $\CS$⟦ \t{[} \p{binders} ⟧ & = & $\CBS$⟦ \p{binders} ⟧  \\       % & (15) \\
      $\CS$⟦ \p{term} ⟧         & = & $\CT$⟦ \p{term } ⟧    \\ [3pt] % & (16) \\
      %
      $\CBS$⟦ \t{VARIABLE}$<$boundvar=x$>$ \p{binders}$<$bound=x$>$ ⟧  & = & \t{[ VARIABLE}$<$boundvar=x$>$ \t{]} $\CBS$⟦ \p{binders} ⟧$<$bound=x$>$ \\% & (17) \\
      $\CBS$⟦ \t{]} \p{term} ⟧  & = & $\CT$⟦ \p{term} ⟧ \\ % & (18)  
      %
      $\CMS$⟦ \t{()}⟧                                              & = & \t{()}                                                             \\       % & (13) \\
      $\CMS$⟦ \t{(} \p{term}$_1$ ...  \t{,} \p{term}$_n$ \t{)}⟧   & = & \t{(} $\CT$⟦ \p{term}$_1$⟧ ... \t{,} $\CT$⟦ \p{term}$_n$⟧ \t{)}  \\ [3pt] % & (14) \\
      \end{tabular}
  \caption{Expanding \Tosca to \Tosca Core.}
  \label{fig:normalization}
\end{figure*}
}

The previous sections built enough of the necessary scaffolding in
order to demonstrate \Tosca is based on solid CRS
foundation. The last stage consists of removing the constraint on bare
necessities to add \emph{syntactic sugar} to make\Tosca programs
more consumable.

To show how \Tosca Core is expanded to \Tosca, we first
need to formalize the subset of the \Tosca syntax introduced in
the Section~\ref{sec:language}. The syntax for defining types has been
left out since the current implementation mostly ignores types. The complete
grammar is available at
%
\anolink{\ToscaPath/src/parser/TransScript.g4}
        {\cd{Tosca.g4}}{\cd{src/parser/TransScript.g4}}.
%
\begin{lstANTLR}
transscript     : decl+										 

decl            : 'rule' CONSTRUCTOR args? → terms
                | 'enum' CONSTRUCTOR    
		
term           : CONSTRUCTOR args?							
                | '(' (term (',' term)*)? ')'
                | VARIABLE
                | STRING 
                | NUMBER  								                                      
                | METAVAR margs?                          
                | concrete												

args            : '(' (scope (',' scope)*)? ')' 
scope           : '[' binders       						
                | term										
binders         : VARIABLE<boundvar=x> binders<bound=x>    
                | ']' term									               
margs           : '(' (term (',' term)*)? ')'
\end{lstANTLR}
%
The equations formalizing the correspondence between \Tosca and
\Tosca Core are shown in Figure~\ref{fig:normalization}. Here,
$\mathcal{C}$ stands for $\mathcal{C}\text{orify}$. The subscripts
indicate the \antlr production in question. To improve readability,
\Tosca constructs are presented on the equation left-hand
side. It is easy to see that all cases from the above grammar are
fully covered. Most of the equations are straightforward and self-explanatory. 
The list syntax is expanded to nested \ToscaIn{Cons} and \ToscaIn{Nil} values.
The main difficulty is normalizing embedded programs, done by $\CC$;  
Section~\ref{sec:normalization} presents this process in details.

%%% Local Variables:
%%% mode: latex
%%% TeX-master: "techreport"
%%% End:
