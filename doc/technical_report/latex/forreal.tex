
\Tosca defines many standard libraries. Most of the functions in the standard
libraries have been written in \Tosca itself. However in the spirit of \Tosca
being light-weight, severals have been written directly in \java. At the \Tosca level, 
only the function signature is specified along with the keyword \ToscaIn{extern} to indicate the
actual implementation is done in \java.
We describe the standard library here briefly:
%
\begin{itemize}
% core
\item \anolink{\ToscaPath/src/std/core.tsc} %
              {Core}{\cd{src/std/Core}} %
              %
              defines basic functionally \ToscaIn{Eval} to force
              evaluation, \ToscaIn{Error}, and
              \ToscaIn{Trace} to trace computation, as well as
              \ToscaIn{Option} and \ToscaIn{Boolean}.
% num
\item \anolink{\ToscaPath/src/std/num.tsc} %
              {Num}{\cd{src/std/Num}} %
              %  
              defines functions on (and external interfaces to) on
              numbers like \ToscaIn{Plus}, \ToscaIn{Minus}
              \ldots
% list
\item \anolink{\ToscaPath/src/std/listdef.tsc} %
              {List}{\cd{src/std/List}} %
              %
              gives list functionalities as expected:
              \ToscaIn{Map} (as described in
              Section~\ref{sec:language}), \ToscaIn{Foldl},
              \ToscaIn{Foldr}, \ToscaIn{Filter}, \ldots
              but also, of course, \ToscaIn{Head},
              \ToscaIn{Tail}, \ToscaIn{Append} \ldots
% map
\item \anolink{\ToscaPath/src/std/mapdef.tsc} %
              {Map}{\cd{src/std/Map}} %
              %
              defines the interfaces for manipulating hash map,
              such as \ToscaIn{MapPut} to add a key-value pair,
              or \ToscaIn{MapKeys} to get the list of keys in a map. 
% pair
\item \anolink{\ToscaPath/src/std/pairdef.tsc} %
              {Pair}{\cd{src/std/Pair}} %
              %
              defines pairs and and functionality on them.
% string
\item \anolink{\ToscaPath/src/std/string.tsc} %
              {String}{\cd{src/std/String}} %
              % 
              defines basic functionalities on Strings such as
              \ToscaIn{UpCase}, \ToscaIn{ConcatString}, 
              and interfaces for external string functions.
\end{itemize}

To ease development, we have developed a \Tosca package for the
\Atom text editor \footnote{cf. \href{https://atom.io/}{atom.io}}.
It provides the basis for syntax highlighting. 
It is available here:
%
\anolink{https://atom.io/packages/language-tosca}
        {\Tosca mode}{\Tosca mode}.

\Tosca is a very young system. Still it has been used successfully
to write a \Tosca compiler in \Tosca itself. Indeed since \Tosca 
is designed to generate compilers, it is natural to use it for itself.
While several existing programming languages are self-hosted (e.g., \Lisp or \Rust),
self-hosting \Tosca adds some extra twists. Thus we want to take
the remainder of this section to describe our experiences with using
and bootstrapping \Tosca.

The following steps describe the bootstrapping procedure.
The first step has been to write a minimal \Tosca compiler using a subset of the \crsx language.
It only makes use of few core features (simple pattern matching, no overlapping rules) available
in both \crsx and \Tosca. This allowed us to have a first compiler $\CONE$ taking \Tosca program
and generating \java code. This compiler still depends on \crsx. The second step was to port the \crsx code to \Tosca,
which was very mechanical due to the high-overlap between the two languages. The third step was to generate a second
stand alone compiler $\CTWO$ using $\CONE$ and \crsx. Finally, self-hosting is truly achieved by generating a third
compiler $\CTHREE$ using $\CTWO$.

One twist came from the use of embedded syntax---because in our case
this was now \Tosca syntax. The main issue is to have multiple
versions of the same language, \Tosca in our case, coexist in
the same Java Virtual Machine (\jvm).
%
For instance, consider this actual source code snippet, which defines
a \Tosca rule:
%
\begin{lstANTLR}
rule NDecl(decl⟦ rule #constructor #args? → #term* ⟧) 
  → ...
\end{lstANTLR}

The \Tosca source code is parsed by the stable 
\Tosca parser, while the embedded program (in \ToscaIn{decl}) is parsed
by the latest \Tosca parser which potentially contains
non-backward compatible modifications, e.g., changing \ToscaIn{'→'} to \ToscaIn{'='}.
The difference with existing approaches is \Tosca parses the
source program and the embedded programs at the same time. Fortunately, since \Tosca
compiles to \java, the typical solution consists of using multiple
class loaders, allowing multiple version of the same classes---parser classes in our case--- to coexist in a single \jvm.

On a final note, it was very rewarding to bootstrap \Tosca. 
By actually using it, we got a very good
feeling for its strengths and conveniences---like the embedded
syntax.

%%% Local Variables:
%%% mode: latex
%%% TeX-master: "techreport"
%%% End:
