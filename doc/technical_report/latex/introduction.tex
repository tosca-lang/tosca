%
The use of transpilers, or source-to-source compilation, has recently
became quite popular especially in the world of \javascript---with
\coffeescript, \babel, and \typescript leading the way.  The reason is
largely due to the fact that the \javascript language has been slow to
evolve and most importantly when new features become available
adoption is slow as older browsers do not support them. Transpilers
solve both problems.
%
Based on this, and partly motivated by the success of transpiler, we
present in this paper \Tosca, a generic compiler generator.

\Tosca aims to combine two different trends in compiler
technology, which have proven to be successful in the past.
%
On one hand, \Tosca follows the lead of transpiling and focuses
on source-to-source transformation. Here we are attracted by the
simplicity and elegance we see in the design of the above mentioned
preprocessing languages for \javascript. In particular, we aspire a
light-weight approach to compiler generation, which focuses on
high-level code generation.
%
On the other hand, we recognize the fundamental role played by
(higher-order) rewriting for various tasks in a compiler. We rewrite
source code to its abstract syntax tree in parsing, we normalize
(i.e., rewrite) the AST to an intermediate representation, we optimize
(i.e., rewrite) the intermediate language in various forms, and
finally we rewrite everything in code generation. It was the
fundamental idea behind \crsx%
\footnote{cf.~\href{http://crsx.org/}{crsx.org}.}  that this informal
use of rewriting can be made precise and even formalized in the
formalism of \emph{Combinatory Reduction Systems} (\emph{CRSs} for
short).
%
We summarize the contributions of the paper.
%
\begin{enumerate}
\item We present a compiler generator that lifts the concept of
  transpilers to its full generality. Given a source language $\PL$,
  whose syntax is represented as an \antlr v4 grammar, and whose
  semantics is provided in a rule-based form, the \Tosca
  executable emits source code in the target language. After employing
  an off-the-shelf compiler for the target language, we obtain an
  executable compiler for $\PL$. Our current setup uses \java 8 as
  target language, but \cpp or \haskell could easily well be
  integrated.

\item For the implementation of \Tosca we have designed a rule-based,
  higher-order, lazy functional language, also called \Tosca.  The
  formal core of \Tosca are CRSs. The foundation in higher-order
  rewriting allows for a good coupling of the tasks of
  source-to-source compilation (parsing, normalizing, generation of
  code) and their implementation. In particular the semantics of the
  source language $\PL$ is provided in \Tosca itself.

\item Finally, \Tosca bootstraps. The compiler for \Tosca
  is generated via the \Tosca executable.
\end{enumerate}

The compiler generator \Tosca and the programming language that
comes along are \emph{light-weight}. As a compiler generator it acts
like a preprocessor and thus does not rely on a full compilation
stack. This entails that code optimization is most of the time passed
on to the target language compiler. Similarly, the \Tosca leaves
a small footprint, as it relies on state-of-the-art features in the
design of the target language.

\Tosca adds features to general purpose functional programming,
some unique and some not, specific to compilers. More precisely
\Tosca features support for \emph{embedded programs},
\emph{enumeration}, \emph{syntactic variables}, and
\emph{specificity-ordered pattern matching}.
%
In particular, the last point is unique. In programming patterns are
typically non-ambiguous as the patterns are not overlapping, or made
non-ambiguous by fixing an order on the defining equations, e.g.,
lexical order. Similarly, pattern matching is paramount in compiler,
where they are used to accomplish dispatch and code optimization. The
latter may even introduce overlapping definitions, as special cases
are set aside against a fallback definition. In complex languages it
becomes increasingly hard to identify missing or redundant cases,
without hampering code optimization.

\emph{Specificity-ordered pattern matching} overcomes this issue to
some degree.  Instead of relying on an ad-hoc lexical order of the
rule definitions to disambiguate, we incorporate Kennaway's
\emph{specificity} rule~\cite{1990_kennaway} and lift it to
higher-order programming, employing higher-order unification for
patterns.  The specificity idea demands that any function definition
can only be employed if there is no alternative rule which is more
specific. The latter can be verified by the generation of a
\emph{specificity tree} that allows to disambiguate between
overlapping definitions. In particular this process allows to
recognize the fallback option easily, as long as we do not have to
disambiguate within equal levels of specificity. In the latter case,
the specificity rule is not applicable and an exception is thrown.

The remainder is structured as follows. In the next section we provide
more background information and give an informal presentation of the
bells and whistles of \Tosca. We also introduce \MiniML as a source
language that will be used as a running example to demonstrate
aforementioned bells and whistles.  In
Section~\ref{sec:theory_practice}, we provides an informal
introduction to CRSs and the translation of this foundation to \Tosca
Core and eventually to the full \Tosca language.
Section~\ref{sec:architecture} describes the architecture underlying
\Tosca.  In Section~\ref{sec:forreal}, we explain how we can use and
used \Tosca---for real. In particular we describe the bootstrapping
process. Finally we conclude and present future work in
Section~\ref{sec:conclusion}.

The paper occasionally refers to the actual code of the implementation
for reference. The reader may find the source (in anonymized form) in
the supplementary material for this submission.

%%% Local Variables:
%%% mode: latex
%%% TeX-master: "techreport"
%%% End:
